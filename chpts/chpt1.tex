\begin{problem}{1}
\end{problem}
\begin{solution} 
    $300\times {10}^{6} \times 0.5 \times \frac{2}{365^3}$
\end{solution}
\begin{problem}{2}
\end{problem}
\begin{solution}
    Suppose $n$ identical balls are thrown into $m$ boxes. Let $\Omega$ be the
    set of $m$-tuples where each entry is a non-negative integer and the sum of
    the entries is $n$. Then, by a stars and bars argument, we know that
    $|\Omega| = \binom{n+m-1}{m-1}$. Therefore, 
    \[
      \Pr{\text{first box is empty}} = \frac{\binom{n+m-2}{m-2}}{\binom{n+m-1}{m-1}} = \frac{m-1}{n+m-1}.
    \]
\end{solution}
\begin{problem}{3} 
\end{problem}
\begin{solution}
    Let $Omega$ be the set of tuples where the first entry ist he number of
    defective items and the second is the number of good items. Then,
    $\Pr{(0,10)} = \frac{\binom{90}{10}}{\binom{100}{100}} = \frac{90\cdot 89
    \cdot \ldots \cdot 81}{100 \cdot 99 \cdot \ldots \cdot 91}$.
\end{solution}
\begin{problem}{4}
\end{problem}
\begin{solution}
    By way of contradiction, suppose that $\Pr{|\xi| > C} > 0$ for $C > 0$.
    Then, it follows that there exists some $A' \geq 1$ such that $\Pr{|\xi|
    \geq A'C} > 0$. So, 
    \begin{align*}
        \Exp{|\xi|^m} &= \sum_{\omega \in \Omega} |\xi(\omega)|^m, \\
        &\geq \sum_{\omega \in \set{|\xi| \geq A'C}} |\xi(\omega)|^m, \\
        &\geq A'^mC^m\Pr{|\xi| \geq A'C}.
    \end{align*}
    Since $\Pr{|\xi| \geq A'C}$ is constant, clearly there does not exist $A >
    0$ such that $A \geq A'^m$ for all $m$, thus a contradiction.
\end{solution}
\newpage
\begin{problem}{5}
\end{problem}
\begin{solution}
    This is the famous hat-check problem. We find the probability that no one
    gets the right letter. Let $\Omega = S_n$, the permutation group of $n$
    elements, also let $C_i$ be the set of all permutations where the
    $i$\textsuperscript{th} element is in the $i$\textsuperscript{th} position.
    Formally, $\sigma \in C_i$ if and only if $\sigma(i) = i$. Then, by the
    inclusion-exclusion principle, we have that 
    \begin{gather*}
        \Pr{\set{\sigma: \exists i, \sigma(i) = i}} = \Pr{\bigcup_{i=1}^n C_i}, \\
        = \sum_{i=1}^n \Pr{C_i} - \sum_{i < j} \Pr{C_i \cap C_j} + \sum_{i < j < k} \Pr{C_i \cap C_j \cap C_k} - \ldots.
    \end{gather*}
    Now, we derive the expression for each term in the previous sum. For any
    $\sigma \in C_{i_1} \cap C_{i_2} \cap \ldots \cap C_{i_k}$, $\sigma(i_1) =
    i_1$, $\sigma(i_2) = i_2$, $\ldots$, $\sigma(i_k) = i_k$. Thus, $\Pr{C_{i_1}
    \cap C_{i_2} \cap \ldots \cap C_{i_k}} = \frac{(n-k)!}{n!}$. Therefore,
    $\sum_{i_1 < i_2 < \ldots < i_k} \Pr{C_{i_1} \cap C_{i_2} \cap \ldots \cap
    C_{i_k}} = \binom{n}{k}\frac{(n-k)!}{n!} = \frac{1}{k!}$. So,
    \begin{align*}
        \Pr{\set{\sigma: \exists i, \sigma(i) = i}} &= 1 - \frac{1}{2!} + \frac{1}{3!} - \ldots + \frac{(-1)^{n+1}}{n!}, \\
        \lim_{n\to\infty} \Pr{\set{\sigma: \exists i, \sigma(i) = i}} &= 1 - \frac{1}{e}.
    \end{align*}
\end{solution}
\begin{problem}{6}
\end{problem}
\begin{solution}
    The first part of the question is a direct application of a stars and bars
    argument. Then, there are $\binom{n+r-1}{r-1}$ solutions to the equation
    $x_1 + \ldots + x_r = n$. By the same argument, we find that 
    \[
        \Pr{x_1 = a} = \binom{n-a+r-2}{r-2} / \binom{n+r-1}{r-1} =  \frac{(r-1)n!(n-a+r-2)!}{(n+r-1)!(n-a)!}.
    \]
    Now, we take the limit as $r,n\to\infty$ and $n/r \to \rho > 0$.
\end{solution}
\begin{problem}{7}
\end{problem}
\begin{solution}
    Recall, that the Poisson distribution is the measure on $\Z^+$ such that for
    any elementary outcome $k$, $\Pr{k} = \frac{\lambda^k}{e^\lambda k!}$. Let
    $\xi = \Id_{\Z^+}$. Then, 
    \begin{align*}
        \Exp{\xi} &= \sum_{n=0}^\infty \xi(n)\Pr{n}, \\
        &= \sum_{n=1}^\infty \frac{\lambda^n}{e^\lambda (n-1)!}, \\
        \intertext{expanding the sum to see the Taylor expansion more clearly, we see that}
        &= \frac{\lambda}{e^\lambda}\left(1 + \lambda + \frac{\lambda^2}{2!} + \ldots \right), \\
        &= \frac{\lambda}{e^\lambda}e^\lambda, \tag{by Taylor expansion}, \\
        &= \lambda.
    \end{align*}
\end{solution}
\begin{problem}{9}
\end{problem}
\begin{solution}
    Recall that if $F$ is a distribution function of a random variable $\xi$
    then, $F_\xi(x) = \Pr{\set{\xi(\omega) < x: \omega \in \Omega}}$. We then evaluate the expression on the right-hand side first 
    \begin{align*}
        \Pr{\set{\omega \in \Omega: \xi(\omega) \leq x}} - \lim_{\delta \downarrow 0} \Pr{\set{\omega \in \Omega: \xi(\omega) \leq x-\delta}} &= \Pr{\set{\omega \in \Omega: \xi(\omega) \leq x}} - \Pr{\bigcup_{i=1}^\infty \set{\omega \in \Omega: \xi(\omega) \leq x - \delta_i}},
    \end{align*}
    where $\delta_1, \delta_2, \ldots$ is any sequence where $\delta_i \geq 0$ and converges to 0. We first show that \[\bigcup_{i=1}^\infty \set{\omega \in \Omega: \xi(\omega) \leq x - \delta_i} = \set{\omega \in \Omega: \xi(\omega) < x},\] 
    That the left-hand side is a subset of the right-hand side is trivial. Now, suppose an element $\omega$ in the set on the right-hand side. There must exist some $\epsilon > 0$, such that $\xi(\omega) = x - \epsilon$. By convergence of $\delta_1,\delta_2,\ldots$, there must exist some $\delta_i$ where $\xi(\omega) = x - \epsilon \leq x - \delta_i$. Thus, the two sets are equivalent. Therefore, 
    \begin{align*}
        \Pr{\set{\omega \in \Omega: \xi(\omega) \leq x}} - \Pr{\bigcup_{i=1}^\infty \set{\omega \in \Omega: \xi(\omega) \leq x - \delta_i}} &= \Pr{\set{\omega \in \Omega: \xi(\omega) \leq x}} - \Pr{\set{\omega \in \Omega: \xi(\omega) < x}}, \\
        &= \Pr{\set{\omega \in \Omega: \xi(\omega) = x}}, \\
        &= \Pr{\xi = x}.
    \end{align*}
    The second equality follows $\sigma$-additivity of $\mathbb{P}$.
\end{solution}
\begin{problem}{10}
\end{problem}
\begin{solution}
    Let $F$ be the distribution of $\xi$ and let $\eta = a\xi + b$ where $a\neq 0$. We first find the distribution of $\eta$:
    \begin{align*}
        \Pr{\set{\omega \in \Omega: \eta(\omega) \leq x}} &= \Pr{\set{\omega \in \Omega: a\xi(\omega) + b \leq x}}, \\
        &= \Pr{\set{\omega\in\Omega:\xi(\omega) \leq \frac{x-b}{a}}}, \\
        &= F\left(\frac{x-b}{a}\right).
    \end{align*}
    Thus, taking the derivative to yield the density function, we have that $p_\eta(x) = \frac{1}{a}p_\xi\left(\frac{x-b}{a}\right)$ where $p_\xi$ is the density of $\xi$. 
\end{solution}
\begin{problem}{11}
\end{problem}
\begin{solution}
    We proceed in the same way as the previous problem, by finding the distribution function of $\eta = \sin \xi$, where $\xi = \Id_{[0,2\pi]}$. Define $\xi' = \frac{1}{2}(\xi - \pi)$. 
    
\end{solution}
\begin{problem}{12}
\end{problem}
\begin{solution}
    
\end{solution}
\begin{problem}{13}
\end{problem}
\begin{solution}
    
\end{solution}
\begin{problem}{14}
\end{problem}
\begin{solution}
    
\end{solution}
\begin{problem}{15}
\end{problem}
\begin{solution}
    
\end{solution}