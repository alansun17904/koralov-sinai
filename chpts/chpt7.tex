\begin{problem}{1}
    Let $y_1, y_2,\ldots$ be a sequence such that $0 \leq y_n \leq 1$ for all $n$, and $\sum_{n=1}^\infty y_n = \infty$. Prove that $\prod_{n=1}^\infty (1-y_n) = 0$.
\end{problem}
\begin{solution}
    Since $\sum_{n=1}^\infty y_n = \infty$, there must exists infinitely many non-zero elements. Consider first the trivial case where there exists $i \in \N$ such that $y_i = 1$, then trivially the claim holds. Otherwise, consider the subsequence $\{y_{i_k}\}_{k=1}^\infty$ where for any $i_k\in\N$, such that $0 < y_{i_k} < 1$. Then, it must be that $\prod_{k=1}^{n} (1-y_{i_k}) > \prod_{k=1}^m (1-y_{i_k})$ for any $m > n$. Thus, the sequence monotonically decreases and by the Squeeze Theorem, the infinite product converges to zero.
\end{solution}

\begin{problem}{2}
    Let $\xi_1,\xi_2,\ldots$ be independent identically distributed random variables. Prove that $\sup_n \xi_n = \infty$ almost surely if and only if $\Pr[\xi_1 > A] > 0$ for every $A$. 
\end{problem}
\begin{solution}
    Suppose $\Pr[\xi_1 > A] > 0$ for every $A$. Then, 
    \begin{align*}
        \Pr[\sup_n \xi_n = \infty] &= 1 - \Pr[\sup_n \xi_n < \infty], \\
        &= 1 - \Pr\left[\bigcup_{A = 1}^\infty \sup_n \xi_n < A \right], \\
        &\geq 1 - \sum_{A=1}^\infty \prod_{k=1}^\infty (1 - \Pr[\xi_1 > A]), \\
        &\geq 1.
    \end{align*}
    The last inequality follows from the previous problem, since we know that for a fixed $A$, $0 < \Pr[\xi_1 > A] \leq 1$ and $\sum_{k=1}^\infty \Pr[\xi_1 > A] = \infty$. Now, by way of contraposition, suppose there exists $A > 0$ such that $\Pr[\xi_1 > A] = 0$. It would suffice to show that $\Pr[\sup_n \xi_n \leq A] = 1$. 
    \begin{align*}
        \Pr[\sup_n \xi_n \leq A] &= \prod_{k=1}^\infty \Pr[\xi_1 \leq A], \tag{by i.i.d.}\\
        &= \prod_{k=1}^\infty (1- \Pr[\xi_1 > A]), \\
        &= 1.
    \end{align*}
    Thus, by contraposition, the claim in the other direction holds.
\end{solution}

\begin{problem}{3}
    Let $\xi_1, \xi_2,\ldots$ be a sequence of random variables defined on the same probability space. Prove that there exists a numeric sequence $c_1,c_2,\ldots$ such that $\xi_n/c_n \to 0$ almost surely as $n\to\infty$.
\end{problem}
\begin{solution}
    Let us first show the following lemma which is a simple consequence of the First Borel-Cantelli Lemma:
    \begin{lemma}
        Let $\xi_1,\xi_2,\ldots$ be random variables defined on the same probability space. Also, let $\epsilon_1,\epsilon_2,\ldots$ be a sequence of non-negative real numbers that converge to zero. If $\sum_{k=1}^\infty \Pr[\xi_k \geq \epsilon_k] < \infty$, then the sequence of random variables converge to 0, almost surely. 
    \end{lemma}
    \begin{proof}
        Define $A \coloneqq \{ \omega \in \Omega: \text{there is an infinite sequence}\,n_i(\omega)\,\text{such that}\,\xi_{n_i}(\omega) \geq \epsilon_{n_i} \}$. Then, given the assumptions, by the First Borel-Cantelli Lemma, $\Pr[A] = 0$. Therefore, the sequence converges almost surely. 
    \end{proof}
    Given this lemma, simply choose $c_k$ such that $\Pr[\eta_k / c_k \geq \epsilon_k] < 1/2^k$. This satisfies the assumptions of the previous lemma and implies the claim. 
\end{solution}

\begin{problem}{4}
    For each $\gamma > 2$, define the set $D_\gamma \subset [0,1]$ as follows: $x \in D_\gamma$ if there is $K_\gamma(x) > 0$ such that for each $q \in \N$
    \[
        \min_{p \in\N} \left|x - \frac{p}{q}\right|  \geq \frac{K_\gamma(x)}{q^\gamma}.
    \]
    Prove that $\lambda(D_\gamma) =1$, where $\lambda$ is the Lebesgue measure on $([0,1], \cB([0,1]))$.
\end{problem}
\begin{solution}
    % outline of the proof

    % bound the probability of a point not being a Diophantine number
        % negate the def of a Diophantine number
        % for each epsilon, create an event
        % bound this event 
        % show that the sum of the probabilities of these events in bounded
        % apply Borel-Cantelli

    Fix $\gamma > 2$. If $x \notin D_\gamma$, then for all $\epsilon > 0$ there exists a $q \in \N$ such that \[
        \min_{p \in \N} \left|x - \frac{p}{q}\right| < \frac{\epsilon}{q^\gamma}.
    \]
    We proceed to show that $\lambda([0,1] \setminus D_\gamma) = 0$ which implies the claim. Let \[
    E_\epsilon = \{x \in [0,1]: \exists q \in \N, \min_{p\in \N} |x-p/q| < \epsilon/q^\gamma\}.
    \]
    Consider the sequence of events $\{E_{1/2^{-k}}\}_{k=1}^\infty$. For a given $x \in [0,1]$, if there exists a subsequence $E_{n_1}, E_{n_2},\ldots$ of the aforeconstructed sequence of events such that $x \in E_{n_i}$ for all $i$, then $x \notin D_\gamma$. Thus, 
    it remains to show that $\Pr[E_k] < \infty$ for fixed $k > 0$. Then, through the First Borel-Cantelli Lemma, we assert that the aforementioned event is measure-zero. 
    \begin{align*}
        \Pr[E_k] &= \Pr\left[\bigcup_{q=1}^\infty \{x \in [0,1]: \min_{p\in \N} |x - p/q| < \epsilon / q^\gamma\}\right], \\
        &\leq \sum_{q=1}^\infty \Pr[\{x \in [0,1]: \min_{p\in\N} |x-p/q| < \epsilon / q^\gamma\}], \\
        &\leq \sum_{q=1}^\infty \Pr[\{x \in [0,1/q]: x \leq \epsilon / q^\gamma \}], \\
        &= \sum_{q=1}^\infty \epsilon / q^{\gamma -1}.
    \end{align*}
    Since $\gamma > 2$, the sequence in the previous equality converges. Let $M \coloneqq \sum_{q=1}^\infty 1/q^{\gamma -1}$. Then, it follows that 
    \[
    \sum_{k = 1}^\infty \Pr[E_{1/2^{-k}}] = \sum_{k=1}^\infty \frac{M}{2^{-k}} = M < \infty. 
    \]
    Therefore, by the First Borel-Cantelli Lemma, the probability that $x\in [0,1]$ appears infinitely often in the events $\{E_{1/2^{-k}}\}_{k=1}^\infty$ is zero. Since we have bounded the negation by above by zero, this suffices in showing the claim.
\end{solution}

\begin{problem}{5}
    Let $\xi_1,\ldots,\xi_n$ be a sequence of $n$ independent random variables, each $\xi_i$ having a symmetric distribution. That is, $\Pr[\xi_i \in A] = \Pr[\xi_i \in -A]$ for any Borel set $A \subset \R$. Assume that $\Exp{\xi_i^{2m}} < \infty$ for $i = 1,2,\ldots,n$. Prove the stronger version of the Kolmogorov inequality:
    \[
        \Pr[\max_{1\leq k \leq n} |\xi_1 +\ldots+\xi_k| \geq t] \leq \frac{\Exp{(\xi_1+\ldots+\xi_n)^{2m}}}{t^{2m}} 
    \]
\end{problem}
\begin{solution}
    Let $C_k = \{\omega \in \Omega : |\xi_1(\omega) + \ldots + \xi_i(\omega) < t\,\text{for}\,1 \leq i <k\,\text{and}\,|\xi_1(x) + \ldots + \xi_k(x) \geq k\}$. Then, it follows that for $i \neq j$, $C_i \cup C_j = \emptyset$ and we have that $C \coloneqq \bigsqcup_{k=1}^n C_k$ is exactly the event that we wish to bound. We start by expanding out the right hand-side of the inequality:
    \begin{align*}
        \Exp{(\xi_1+\ldots+\xi_n)^{2m}} &= \int_\Omega (\xi_1+\ldots+\xi_n)^{2m}~d\PP, \\
        &\geq \sum_{k=1}^n \int_{C_k} (\xi_1 + \ldots + \xi_n)^{2m}~d\PP, \\
        &= \sum_{k=1}^n \int_{C_k}\left[((\xi_1 + \ldots \xi_k) + (\xi_{k+1}+\ldots+\xi_n))^2\right]^m~d\PP, \\
        &= \sum_{k=1}^n \int_{C_k} \left[(\xi_1 + \ldots + \xi_k)^2 + 2(\xi_1+\ldots+\xi_k)(\xi_{k+1} + \ldots \xi_n) + (\xi_{k+1}+\ldots + \xi_n)^2\right]^m~d\PP, \\
        &\geq \sum_{k=1}^n \int_{C_k}\left[t^2 + 2\cdot \sum_{\substack{1 \leq i \leq k \\ k+1\leq j \leq n}} \xi_i\xi_j\right]^m~d\PP.
    \end{align*}
    The last inequality follows from the construction of $C_k$ and the fact that the term $(\xi_{k+1}+\ldots+{\xi_n})^2$ is non-negative. It remains to bound the integrand:
    \begin{align*}
        \int_{C_k}\left[t^2 + 2\sum_{\substack{1 \leq i \leq k \\ k+1\leq j \leq n}} \xi_i\xi_j\right]^m~d\PP &= \int_{C_k} \sum_{n = 0}^m  \binom{m}{n} (t^2)^{m-n} \left(2\cdot \sum_{\substack{1 \leq i \leq k \\ k+1\leq j \leq n}} \xi_i\xi_j \right)^n~d\PP, \\
        &= \sum_{n=0}^m \binom{m}{n}  (t^2)^{m-n} \int_{C_k}2^n\cdot \left(\sum_{\substack{1 \leq i \leq k \\ k+1\leq j \leq n}} \xi_i\xi_j \right)^n~d\PP.
    \intertext{Now, consider the summation under the integrand. By independence and symmetry $\Exp{\xi_i^p \xi_j^q} = 0$ if either $p$ or $q$ is odd, and if both $p,q$ are even, then the expectation must be non-negative. Therefore, continuing}
        &\geq \sum_{k=1}^n \int_{C_k} t^{2m}~d\PP = t^{2m}\Pr[C].
    \end{align*}
    Thus, by the construction of $C$, the claim is shown.
\end{solution}

\begin{problem}{6}
    Let $\xi_1,\xi_2,\ldots$ be independent random variables with non-negative values. Prove that the series $\sum_{i=1}^\infty \xi_i$ converges almost surely if and only if \[
    \sum_{i=1}^\infty \Exp{\frac{\xi_i}{1 + \xi_i}} < \infty.
    \]
\end{problem}
\begin{solution}
    Suppose that $\sum_{i=1}^\infty \xi_i$ converges almost surely. Then, it must be that $\Exp{\sum_{i=1}^\infty \xi_i}$ converges which implies that $\sum_{i=1}^\infty \Exp{\xi_i}$ converges. Since $\xi_i$ is non-negative, $\frac{\xi_i}{1+ \xi_i} < \xi_i$ everywhere, the claim in the forward direction holds. Now, suppose that $\sum_{i=1}^\infty \Exp{\frac{\xi_i}{1 + \xi_i}} < \infty$, we wants to show that $\sum_{i=1}^\infty \xi_i$ converges almost surely. Since the random variables non-negative, it suffices to show that $\sum_{i=1}^\infty \xi_i < \infty$ almost surely.
\end{solution}

\begin{problem}{7}
    Let $\xi_1,\xi_2,\ldots$ be a sequence of independent identically distributed random variables with uniform distribution on $[0,1]$. Prove that the limit \[
    \lim_{n\to\infty} \sqrt[n]{\xi_1\cdot \ldots \cdot \xi_n}
    \]
    exists with probability one. Find its value.
\end{problem}
\begin{solution}
To show that $\lim_{n\to\infty} \sqrt[n]{\xi_1,\cdot \ldots \cdot \xi_n}$ converges almost surely, it suffices to show that \[
\exp\left(\lim_{n\to\infty} \log\left(\sqrt[n]{\xi_1,\cdot \ldots \cdot \xi_n}\right)\right)
\] converges almost surely. Thus, we evaluate this limit. Since 
\[
\exp\left(\lim_{n\to\infty} \log\left(\sqrt[n]{\xi_1,\cdot \ldots \cdot \xi_n}\right)\right) = \exp\left(\lim_{n\to\infty} \frac{1}{n}\sum_{k=1}^n\log\xi_k\right),
\]
we first find $\Exp{\log(\xi_i)}$, then seek to apply the strong law of large numbers. So,
\begin{align*}
    \Exp{\log(\xi_i)} &= \int_0^1 \log(x)~dx, \\
    &= x\log(x) - x \big\vert_0^1 = -1. 
\end{align*}
Since $\log(\xi_i)$ has finite mathematical expectation and the sequence of random variables $\xi_i$ are independent and identically distributed by Second Kolmogorov Theorem, it follows that 
\[
\exp\left(\lim_{n\to\infty} \frac{1}{n}\sum_{k=1}^n\log\xi_k\right) = e^{-1},
\]
almost surely. 
\end{solution}

\begin{problem}{8}
\end{problem}
\begin{solution}
    
\end{solution}