\begin{problem}{1}
    Let $f_n, n\geq 1$, and $f$ be measurable functions on a measurable space $(\Omega,\cF)$. Prove that the set $\{\omega: \lim_{n\to\infty} f_n(\omega) = f(\omega)\}$ is $\cF$-measurable.
\end{problem}
\begin{solution}
    It suffices to show that 
    \begin{equation}\label{eq:limit-sig-alg}
        \{\omega \in \Omega: \lim_{n\to\infty} f_n(\omega) = f(\omega)\} = \bigcap_{i=1}^\infty \bigcup_{j=1}^\infty \bigcap_{k=j}^\infty \left\{\omega \in \Omega : |f_k(\omega) - f(\omega)| < \frac{1}{i}\right\}.
    \end{equation}
    This is because
    \[
        \left\{\omega \in \Omega : |f_k(\omega) - f(\omega)| < \frac{1}{i}\right\} =  (f_k - f)^{-1}\left(-\frac{1}{i},\frac{1}{i}\right).
    \]
    Since both $f_k$ and $f$ are $\cF$-measurable functions, then $f_k - f$ is also a measurable function. Moreover, since $(-1/i, 1/i)$ is clearly a Borel set, then $(f_k - f)^{-1}(-1/i, 1/i) \in \cF$. It follows from the axioms of a $\sigma$-algebra that the right hand-side of Eq.~\ref{eq:limit-sig-alg} is also a set in $\cF$. 

    Now, we show Eq.~\ref{eq:limit-sig-alg}:
    \begin{align*}
        x \in \bigcap_{i=1}^\infty \bigcup_{j=1}^\infty \bigcap_{k=j}^\infty \left\{\omega \in \Omega : |f_k(\omega) - f(\omega)| < \frac{1}{i}\right\} &\Rightarrow x \in \bigcup_{j=1}^\infty\bigcap_{k=j}^\infty \left\{\omega \in \Omega : |f_k(\omega) - f(\omega)| <  \frac{1}{i}\right\}, \tag{$\forall i \in \N$} \\
        &\Rightarrow \exists j, \forall k > j, x\in \left\{\omega \in \Omega : |f_k(\omega) - f(\omega)| <  \frac{1}{i}\right\}, \tag{$\forall i \in \N$}, \\
        &\Rightarrow \forall i \in \N, \exists j, \forall k > j, |f_k(x) - f(x)| < \frac{1}{i}.
    \end{align*}
    It follows that $f_1,f_2,\ldots$ to $f$ at $x$. The other direction of the inclusion is trivial and its proof is omitted as it results from the same logic above.
\end{solution}

\begin{problem}{2}
    Prove that if a random variable $\xi$ taking non-negative values is such that \[
        \Pr[\xi \geq n] \geq \frac{1}{n},
    \]
    for all $n\in\N$. Then, $\Exp{\xi} = \infty$.
\end{problem}
\begin{solution}
    \begin{align*}
        \Exp{\xi} &\geq \sum_{k=1}^\infty k\cdot \Pr[k \leq \xi < k + 1], \\
        &= \Pr[1 \leq \xi < 2] + 2\Pr[2 \leq \xi < 3] + 3\Pr[3 \leq \xi < 4] + \ldots, \\
        &= \left(\sum_{k=1}^\infty \Pr[k \leq \xi < k+1]\right) + \Pr[2\leq \xi < 3] + 2\Pr[3 \leq \xi < 4] + \ldots, \\
        &= 1 + \left(\sum_{k=2}^\infty \Pr[k \leq \xi < k+1]\right) + \ldots, \\
        &= 1 + \frac{1}{2} + \frac{1}{3} + \ldots, \\
        &\geq \infty.
    \end{align*}
\end{solution}
Note that we just proved that if $\xi$ is a random variable that only takes on positive integers, then $\Exp{\xi} = \sum_{i=1}^\infty \Pr[\xi \geq i]$.

\begin{problem}{3}
    Construct a sequence of random variables $\xi_n$ such that $\xi_n(\omega) \to 0$ for every $\omega$, but $\Exp{\xi_n}\to\infty$ as $n\to\infty$.
\end{problem}
\begin{solution}
    Let $\xi_n$ be a normally-distributed random variable parameterized by mean $n$ and variance $1$. Then, it is clear that $\xi_n$ converges pointwise to the random variable $\xi(\omega) = 0$, however, 
    \[
        \lim_{n\to\infty} \Exp{\xi_n} = \lim_{n\to\infty} n = \infty.
    \]
\end{solution}

\begin{problem}{4}
A random variable $\xi$ takes values in the interval $[A,B]$ and $\Var{\xi} = \left(\frac{B-A}{2}\right)^2$. Find the distribution of $\xi$.
\end{problem}
\begin{solution}
    Let $\xi$ be a random variable such that $\Pr[\xi = A] = \Pr[\xi=B] = \frac{1}{2}$. Then it follows that 
    \begin{align*}
        \Var{\xi} &= \Exp{\xi^2} - \Exp{\xi}^2, \\
        &= \frac{1}{2}A^2 + \frac{1}{2}B^2 - \left(\frac{1}{2}A + \frac{1}{2}B\right)^2, \\
        &= \frac{1}{2}A^2 + \frac{1}{2}B^2 - \frac{1}{4}\left(A + B\right)^2, \\
        &= \frac{1}{4}A^2 - \frac{1}{2}AB + \frac{1}{4}B^2, \\
        &= \left(\frac{B-A}{2}\right)^2.
    \end{align*}
    Interestingly, this is the maximum variance a bounded random variable can achieve (see \href{https://en.wikipedia.org/wiki/Popoviciu%27s_inequality_on_variances}{Popoviciu's Inequality on Variances}).
\end{solution}

\begin{problem}{5}
Let $\{x_1, x_2,  \ldots\}$ be a collection of rational points from the interval $[0, 1]$. A random variable $\xi$ takes the value $x_n$ with probability $\frac{1}{2^n}$. Prove that the distribution function $F_\xi(x)$ of $\xi$ is continuous at every irrational point $x$.
\end{problem}
\begin{solution}
    Let $x$ be an arbitrary irrational number and also let $\epsilon > 0$. Then, choose $N\in\N$ such that $\log_2(1/\epsilon) + 1 < N$. Then, let 
    \[
        \delta = \min\left\{|x - x_k|\right\}_{k=1}^N.
    \]
    So, for any $x'$ such that $|x - x'| < \delta$, it must be that 
    \begin{align*}
        \left|F_\xi(x) - F_\xi(x')\right| &= \left|\sum_{x_j\,\text{s.t.}\,x_j<x} \frac{1}{2^j} - \sum_{x_j\,\text{s.t.}\,x_j<x'} \frac{1}{2^j}\right|, \\
        &= \sum_{x_j\,\text{s.t.} x' \leq x_j < x} \frac{1}{2^j}.
        \intertext{This equality comes from assuming, without the loss of generality, $x' < x$. So, then}
        &\leq \sum_{j=N}^\infty \frac{1}{2^j}, \\
        &= \frac{1}{2^{N-1}}, \\
        &< \epsilon. \tag{by def.}
    \end{align*}
    Thus, $F_xi(\omega)$ is continuous for irrational $\omega$. 
\end{solution}

\begin{problem}{6}
Let $\xi$ be a random variable with continuous density $p_\xi$ such that $p_\xi(0) > 0$. Find the density of $\eta$, where 
\begin{equation*}
    \eta(\omega) = \begin{cases}
        \frac{1}{\xi(\omega)} &\text{if}\,\xi(\omega) \neq 0, \\
        0&\text{otherwise}.
    \end{cases}
\end{equation*}
Prove that $\eta$ does not have finite expectation.
\end{problem}
\begin{solution}
    We first find the density of $\eta$. Consider first the distribution of $\eta$:
    \begin{align*}
        \Pr[\eta \leq x] &= \Pr\left[\frac{1}{\xi} \leq x\right], \\
        &= 1 - \Pr\left[\xi < \frac{1}{x}\right], \\
        &= 1 - F_\xi\left(\frac{1}{x}\right).
    \end{align*}
    Therefore, the density function of $\eta$, $p_\eta = \left(1-F_\xi\left(\frac{1}{x}\right)\right)' = \frac{1}{x^2}p_\xi\left(\frac{1}{x}\right)$. Now, we apply the definition of the expected value of $\xi$:
    \begin{align*}
        \Exp{\eta} &= \int_{-\infty}^\infty \frac{1}{x}p_\xi\left(\frac{1}{x}\right)~dx, \\
        \intertext{make the substitution for $u = \frac{1}{x}$, then $du=-\frac{1}{x^2}$ and also apply the definition of the improper integeral. So, let $L,R \gg 0$, then}
        &= \lim_{L\to\infty} \int_{1/L}^\infty -\frac{1}{u}p_\xi(u)~du + \lim_{R\to\infty}\int_0^{-1/R} \frac{1}{u}p_\xi(u)~du. 
    \end{align*}
    Since $p_\xi(0) > 0$ and $p_\xi$ is continuous, there exists $a > 0$ such that $p_\xi([0,a)) \subset \R^+$ and $p_\xi((-a,0]) \subset \R^+$. Thus, it follows from the definition of the Lebesgue integral that both 
    \begin{align*}
        \lim_{L\to\infty} \int_{1/L}^\infty -\frac{1}{u}p_\xi(u)~du &\leq \lim_{L\to\infty} -\frac{1}{1/L} \Pr[[0,a)], \\
        \lim_{R \to\infty} \int_0^{-1/R} \frac{1}{u} p_\xi(u)~du &\geq \lim_{R\to\infty} \frac{1}{1/R} \Pr[(-a,0]],  
    \end{align*}
    Since the integrals go to $-\infty$ and $\infty$, respectively, the expected value is then undefined. 
\end{solution}

\begin{problem}{7}
\end{problem}
\begin{solution}
     
\end{solution}

\begin{problem}{8}
Prove that if a sequence of measurable functions $f_n$ converges to $f$ almost surely as $n\to\infty$, then it also converges to $f$ in measure. If $f_n$ converges to $f$ in measure, then there is a subsequence $f_{n_k}$ which converges to $f$ almost surely as $k\to\infty$.
\end{problem}
\begin{solution}
    We first show that convergence almost surely implies convergence in measure by contraposition. We proceed by contraposition. Suppose that a sequence of functions does not convergence in measure. Then, there exists $\epsilon > 0$ such that $\Pr[|f_n - f| > \epsilon] \not \to 0$ as $n\to \infty$. This implies that there exists $B \subset X$ such that $\Pr[X \setminus B] > 0$ and for all $N \in \N$ there exists $n > N$ where $|f_n(x) - f(x)| > \epsilon$ where $x \in B$. Thus, the sequence cannot converge almost surely. 

    It remains to show the partial converse. Let $\{\epsilon_i\}_{i=1}^\infty$ such that $\epsilon_i = 1/i$. Then, for each $\epsilon$, choose $n$ such that \[
        \Pr[\{x \in X: |f_N(x) - f(x)| > \epsilon\}] < \frac{1}{2^j},
    \]
    for all $N > n$. Such an $n$ must exist by the definition of convergence in measure, denote this specific $\epsilon$-dependent $n$ as $n_{\epsilon_i}$. Now construct a subsequence of $\{f_n\}_{n=1}^\infty$, $\{f_{n_k}\}_{k=1}^\infty$ such that $n_k = \max (n_{\epsilon_i}, n_{k-1})+1$ and $n_1 = n_{\epsilon_i}$.

    It remains to show that this subsequence converges almost surely to $f$. Denote by $E_k$ the following set 
    \[
        E_k = \left\{x \in X: |f_{n_k}(x) - f(x)| > \frac{1}{k}\right\}.
    \]
    By construction, $\sum_{k=1}^\infty \Pr[E_k] < \infty$. By the Borel-Cantelli lemma, the probability that there exists $x$ in infinitely many of $E_k$ is 0. More formally, 
    \begin{align*}
        \Pr\left[\bigcap_{n=1}^\infty \bigcup_{k=n}^\infty E_k\right] &= 0, \\
        &= \Pr\left[\bigcap_{n=1}^\infty\bigcup_{k=n}^\infty \left\{x \in X: |f_{n_k}(x) - f(x)| > \frac{1}{k}\right\}\right], \\
        &= \Pr\left[\left\{x \in X: \forall N \exists k > N\,\text{such that}\, |f_{n_k}(x) - f(x) | > \frac{1}{k}\right\}\right].
    \end{align*}
    To complete the proof, we need to show that the set above is a superset of the set of all points that do not converge pointwise. If $x \in X$ does not converge pointwise, then there exists $\epsilon > 0$ such that for all $N \in \N$, there exists $n > N$ where $|f_{n_n}(x) - f(x)| > \epsilon$. Denote this $n$ as $n_\epsilon$. Thus, if $\epsilon < \frac{1}{n_\epsilon}$, then let $n$ be chosen to be $\frac{1}{\epsilon}$. Otherwise, let $n$ be equal to $n_\epsilon$. Then, it follows that $\forall N$, $|f_{n_n} - f(x)| > \frac{1}{n}$. Therefore, the set of points that do not converge pointwise has probability zero. 
\end{solution}

\begin{problem}{9}
Let $F(x)$ be a distribution function. Compute $\int_{-\infty}^\infty F(x+10) - F(x)~dx$.
\end{problem}
\begin{solution}
    Apply integration by-parts:
    \begin{align*}
        \int_{-\infty}^\infty F(x+10) - F(x)~dx &= x(F(x+10) - F(x)) \Big\vert_{-\infty}^\infty - \int_{-\infty}^\infty x(p(x+10) - p(x))~dx, 
    \end{align*}
    where $p(x)$ is the density function corresponding to $F(\cdot)$. First, we evaluate the term on the left, since $lim_{x\to\infty} F(x) = 1$ and $\lim_{x\to-\infty} F(x) = 0$, it must be that $x(F(x+10) - F(x)) \Big\vert_{-\infty}^\infty = 0$. Therefore, it suffices to evaluate the term on the right, 
    \begin{align*}
        -\int_{-\infty}^\infty x(p(x+10) - p(x))~dx &= -\int_{-\infty}^\infty xp(x+10) - xp(x)~dx, \\
        &= -\int_{-\infty}^\infty (x+10)p(x+10) - xp(x)~dx - \int_{-\infty}^\infty 10p(x+10)~dx, \\
        &= -\int_{-\infty}^\infty up(u)~du - \int_{-\infty}^\infty xp(x) + 10, \\
        &= 10.
    \end{align*}
    In general, it follows that for any $u \in \R$, $\int F(x + u) - F(x)~dx = u$ through the same calculation that we just performed. 
\end{solution}

\begin{problem}{10}
\end{problem}
\begin{solution}
\end{solution}

\begin{problem}{11}
\end{problem}
\begin{solution}
\end{solution}

\begin{problem}{12}
Prove the Hölder Inequality.
\end{problem}
\begin{solution}
    We state Young's inequality here without proof. This is crucial to proving Hölder's ineqaulity.
    \begin{theorem}[Young's inequality]
        Suppose $1 < p < \infty$. Then, 
        \[
            ab \leq \frac{a^p}{p} + \frac{b^{q}}{q},  
        \]
        where $q$ is the dual exponent of $p$, for all $a, b \geq 0$.
    \end{theorem}
    Now, we prove Hölder's inequality:
    Suppose first that $p=1,q=\infty$. Then, it follows that 
    \begin{align*}
        \int |fg|~d\mu &\leq \int \lVert g \rVert_\infty |f|~d\mu, \\
        &\leq \lVert g \rVert_\infty  \int |f|~d\mu, \\
        &\leq \lVert f \rVert_1 \lVert g \rVert_\infty.
    \end{align*}
    The same argument can be repeated by bounding the Lebesgue integral of $f$ above by the essential supremum. Now, consider the case where $p > 1$. For any $x\in X$, where $X$ sample space, Young's inequality holds for $p > 1$: 
    \begin{align*}
        \left| \frac{f(x)}{\lVert f \rVert_p} \cdot \frac{g(x)}{\lVert g \rVert_q} \right| &\leq \frac{|f|^p}{p\lVert f\rVert_p} + \frac{|g|^q}{q\lVert g\rVert_q}, \\
        \frac{1}{\lVert f\rVert_p \lVert g \rVert_q}\int |fg|~d\mu &\leq \int \frac{|f|^p}{p\lVert f\rVert_p} + \frac{|g|^q}{q\lVert g\rVert_q}~d\mu, \\
        &= \frac{1}{p} \int \frac{|f|^p}{\lVert f\rVert_p}~d\mu + \frac{1}{q} \int \frac{|g|^q}{\lVert g\rVert_q}~d\mu, \\
        &= \frac{1}{p} + \frac{1}{q}, \\
        \int |fg|~d\mu &\leq \lVert f \rVert_p \lVert g \rVert_q.
    \end{align*}
\end{solution}

\begin{problem}{13}
\end{problem}
\begin{solution}
\end{solution}