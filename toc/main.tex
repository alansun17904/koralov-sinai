\documentclass{article}
\usepackage{lipsum}
\usepackage{ifthen}
\usepackage{tikz-cd}
\usepackage{soul}
\usepackage{fullpage, prettyref}
\usepackage{fancyhdr}
\usepackage{palatino}
\usepackage[T1]{fontenc}
\usepackage{ulem}
\usepackage{graphicx}
\usepackage{ifthen}
\usepackage{algorithm}
\usepackage{a4}
\usepackage[margin=1in]{geometry} 
\usepackage[noend]{algpseudocode}
\usepackage{amsmath, amssymb, amsthm, mathtools}
\usepackage{setspace}
\usepackage{multicol}
\usepackage{array}
\usepackage{enumitem}
\usepackage{csquotes}
\usepackage{url}
\usepackage{boxedminipage}
\usepackage{wrapfig}
\usepackage{color, xcolor}
\usepackage{transparent}
\usepackage{parskip}
\usepackage{varwidth}
\usepackage{bm}
\usepackage{xspace}
\usepackage{url}
\usepackage{fullpage, prettyref}
\usepackage{boxedminipage}
\usepackage{wrapfig}
\usepackage{ifthen}
\usepackage{framed}
\usepackage{mdframed}
\usepackage[colorlinks=true,urlcolor=blue,linkcolor=blue,citecolor=violet,pdfstartview=FitH]{hyperref}
\usepackage[noabbrev,nameinlink]{cleveref}
\usepackage{enumitem}
\usepackage{tikz}
\usepackage{mathrsfs} 
\usepackage{bbm} % bf numbers
% use for citations.
\usepackage[backend=biber,bibencoding=ascii,style=alphabetic,sorting=none]{biblatex}

% config
\renewcommand{\headrulewidth}{0pt}
\renewcommand{\footrulewidth}{0pt}
\pagestyle{fancy}
\fancyhead{}
\fancyfoot{}
\fancyfoot[R]{\thepage}


% Algebra
\newcommand{\unit}[1]{#1^{\times}}
\DeclareMathOperator{\Orb}{Orb}
\DeclareMathOperator{\Stab}{Stab}
\DeclareMathOperator{\Hom}{Hom}
\DeclareMathOperator{\End}{End}
\DeclareMathOperator{\Aut}{Aut}
\DeclareMathOperator{\Gal}{Gal}
\DeclareMathOperator{\rank}{rk}
\DeclareMathOperator{\img}{img}
\DeclareMathOperator{\id}{id}
\DeclareMathOperator{\ch}{ch}
\DeclareMathOperator{\tr}{tr}
\DeclareMathOperator{\codim}{codim}
\DeclareMathOperator{\ann}{ann}
\DeclareMathOperator{\ev}{ev}
\DeclareMathOperator{\GL}{GL}
\DeclareMathOperator{\Ob}{Ob}
\newcommand{\category}[1]{\mathsf{\underline{#1}}}
\newcommand{\acts}{\circlearrowleft}
\newcommand{\tleq}{\unlhd}

% Boolean logic/algorithms
\newcommand{\cor}{\textsc{ or }}
\newcommand{\cand}{\textsc{ and }}
\newcommand{\xor}{\oplus}
\newcommand{\true}{\textsc{True}}
\newcommand{\false}{\textsc{False}}
\newcommand{\nil}{\textsc{Nil}}
\newcommand{\alg}[1]{\textsc{#1}}
\newcommand{\eq}{\leftarrow}
\newcommand{\chapterref}[2]{[\S #1, #2]}

% math formatting
\def\bar{\overline}
\newcommand{\script}[1]{\mathcal{#1}}
\newcommand{\fancyscript}[1]{\mathscr{#1}}


% number sets
\newcommand{\N}{\mathbb{N}}
\newcommand{\Q}{\mathbb{Q}}
\newcommand{\R}{\mathbb{R}}
\newcommand{\Z}{\mathbb{Z}}
\newcommand{\C}{\mathbb{C}}
\newcommand{\F}{\mathbb{F}}

\newcommand{\PP}{\mathbb{P}}

%Paired Delims
\def\set#1{\left\{ #1 \right\}}
\def\angle#1{\left\langle #1 \right\rangle}
\def\sqbrackets#1{\left[ #1 \right]}
\def\brackets#1{\left( #1 \right)}
\DeclarePairedDelimiter\ceil{\lceil}{\rceil}
\DeclarePairedDelimiter\floor{\lfloor}{\rfloor}
\newcommand{\abs}[1]{\left\lvert #1 \right\rvert}

%problem environment
\newenvironment{problem}[2][Problem]{\large \begin{trivlist}
\item[\hskip \labelsep {\bfseries #1}\hskip \labelsep {\bfseries #2.}]}{\end{trivlist}\vspace{1mm}\hline}

%problems list environment
\newlist{subprob}{enumerate}{1}
\setlist[subprob]{label={\bfseries\itshape\alph*.}}

%solution environment
\newenvironment{solution}[1][Solution]{
    \setcounter{nlem}{0}
    \vspace{-2ex}
  \begin{proof}[#1]
}{
  \phantom\qedhere\end{proof}
}


% colored boxes
\newenvironment{answerbox}{
      \begin{mdframed}[backgroundcolor=green!30, innertopmargin=12px, topline=false,rightline=false,leftline=false,bottomline=false]}{
  \end{mdframed}%
}

\newenvironment{thmbox}{
        \vskip2ex
          \begin{mdframed}[backgroundcolor=blue!10, topline=false,rightline=false,leftline=false,bottomline=false]}{
  \end{mdframed}
}

\newenvironment{algbox}[1]{
        \vskip2ex
            \begin{mdframed}[backgroundcolor=yellow!10, topline=false,rightline=false,leftline=false,bottomline=false, userdefinedwidth=.7\linewidth, align=center]
              \vspace{1ex}
              \hrulefill
              \vspace{-1.5ex}
  
              {\bf #1}
  
              \vspace{-2.5ex}
              \hrulefill
              }{
            \end{mdframed}
}

\newenvironment{algboxnt}{
        \vskip2ex
            \begin{mdframed}[backgroundcolor=yellow!10, topline=false,rightline=false,leftline=false,bottomline=false, userdefinedwidth=.7\linewidth, align=center]\vspace{1.5ex}}{
            \end{mdframed}
}

% Caligraphic caps
\newcommand{\cA}{\mathcal{A}}
\newcommand{\cB}{\mathcal{B}}
\newcommand{\cC}{\mathcal{C}}
\newcommand{\cD}{\mathcal{D}}
\newcommand{\cE}{\mathcal{E}}
\newcommand{\cF}{\mathcal{F}}
\newcommand{\cG}{\mathcal{G}}
\newcommand{\cI}{\mathcal{I}}
\newcommand{\cL}{\mathcal{L}}
\newcommand{\cK}{\mathcal{K}}
\newcommand{\cO}{\mathcal{O}}
\newcommand{\cP}{\mathcal{P}}
\newcommand{\cS}{\mathcal{S}}
\newcommand{\cT}{\mathcal{T}}
\newcommand{\cU}{\mathcal{U}}
\newcommand{\cX}{\mathcal{X}}
\newcommand{\cY}{\mathcal{Y}}
\newcommand{\cZ}{\mathcal{Z}}

% Script caps
\newcommand{\sA}{\mathscr{A}}
\newcommand{\sB}{\mathscr{B}}
\newcommand{\sC}{\mathscr{C}}
\newcommand{\sD}{\mathscr{D}}
\newcommand{\sE}{\mathscr{E}}
\newcommand{\sF}{\mathscr{F}}
\newcommand{\sG}{\mathscr{G}}
\newcommand{\sI}{\mathscr{I}}
\newcommand{\sL}{\mathscr{L}}
\newcommand{\sP}{\mathscr{P}}
\newcommand{\sS}{\mathscr{S}}
\newcommand{\sT}{\mathscr{T}}
\newcommand{\sU}{\mathscr{U}}
\newcommand{\sX}{\mathscr{X}}
\newcommand{\sY}{\mathscr{Y}}
\newcommand{\sZ}{\mathscr{Z}}

% theorem envs
\theoremstyle{definition}
\newtheorem*{lem}{Lemma}
\newtheorem*{claim}{Claim}
\newtheorem*{thm}{Theorem}
\newtheorem*{defn}{Definition}
\newtheorem*{coro}{Corollary}
\newtheorem*{rmk}{Remark}
\newtheorem*{prop}{Proposition}
\newtheorem*{example}{Example}
\newtheorem*{probstatement}{Problem}
\newtheorem{nlem}{Lemma}
\newtheorem{ndefn}[nlem]{Definition}
\newtheorem{nthm}[nlem]{Theorem}
\newtheorem{ncor}[nlem]{Corollary}
\newtheorem{nprop}[nlem]{Proposition}
\newtheorem{nrmk}[nlem]{Remark}


% Probability
\newcommand{\Prob}[1]{\mathbb{P}\left[#1\right]}
\newcommand{\Exp}[1]{\mathbb{E}\left[#1\right]}
\DeclareMathOperator{\Var}{Var}
\newcommand{\Cov}[1]{\mathbf{Cov}[#1]}
\renewcommand{\Pr}{\mathbb{P}}
\DeclareMathOperator{\Bin}{Bin}
\DeclareMathOperator{\Pois}{Pois}
\DeclareMathOperator{\Geom}{Geom}
\DeclareMathOperator{\Normal}{\mathcal{N}}
\DeclareMathOperator{\Id}{Id}


% Misc commands
\newcommand{\sgn}{\mathrm{sgn}}
\newcommand{\poly}{\mathrm{poly}}
\newcommand{\polylog}{\mathrm{polylog}}
\newcommand{\contradiction}{\hfill$\Rightarrow\!\Leftarrow$}
\newcommand{\writer}{Alan Sun}
\newcommand{\email}{alan.w.sun.24@dartmouth.edu}
\newcommand{\TODO}[1]{\textcolor{red}{\bf [TODO: #1] }}
\newcommand{\note}[1]{\textcolor{blue}{\bf [Note: #1] }}
\newcommand{\alert}[1]{\textcolor{red}{\bf [#1] }}
\renewcommand{\epsilon}{\varepsilon}
\renewcommand{\phi}{\varphi}
\newcommand{\lnorm}[2]{\left\lVert#1\right\rVert_{#2}}
\DeclareMathOperator*{\argmax}{arg\,max}
\DeclareMathOperator*{\argmin}{arg\,min}


% PSet Titles
\newcommand{\normaltitle}[3]{
    \begin{center}
        {\LARGE #2}
        \vskip-1ex
        {\large \writer}
        \vskip-1ex
        {\large Date: \today}
        \vskip-1ex
        {\large #1~#3}
        \vskip0.1ex
        \vspace{2ex}
        \hrule height 2pt
        \vspace{2ex}
    \end{center}
}

\newcommand{\fancytitle}[3]{
	{
            \large
            \rule{\textwidth}{1pt}
		\begin{center}
			{
                    \setlength{\parskip}{0mm} \setlength{\parindent}{0mm}
				{\textbf{#2} \hfill \writer}\\
				#1:~#3 \hfill Date: \today
                }
		\end{center}
            \vskip-2ex
            \rule{\textwidth}{1pt}
	}
 }
\everymath{\displaystyle}
\usepackage{enumitem}
\usepackage{comment}
\usepackage{bbding}
\usepackage{pifont}
\usepackage{fontawesome}
\usepackage[weather]{ifsym}
\usepackage{hyperref}

\usepackage{xcolor}
\definecolor{brickred}{rgb}{0.8, 0.25, 0.33}
\definecolor{babyblue}{rgb}{0.54, 0.81, 0.94}
\definecolor{arsenic}{rgb}{0.23, 0.27, 0.29}
\definecolor{arylideyellow}{rgb}{0.91, 0.84, 0.42}

\newcommand{\cold}{{\color{babyblue}{\Snowflake}}}
\newcommand{\elegant}{{\color{brickred}{\ding{46}}}}
\newcommand{\prac}{{\color{arsenic}{\ding{43}}}}
\newcommand{\know}{{\color{arylideyellow}{\Lightning}}}


\begin{document}

\fancytitle{\textit{Theory of Probability and Random Processes}}{Need to Knows}{Koralov and Sinai}
Here are a list of things, both theorems and problems, that I think are important to know how to do by heart. Therefore, they warrant frequent review.

\noindent\fbox{%
    \parbox{\textwidth}{%
        \textbf{Key}
        \par \cold: Things that I need to know off the top of my head. 
        \par \prac: Recall the result off the top of my head and if you gave me 15 minutes, I could give you a formal proof or explanation.
        \par \elegant: Elegant results that serve as great examples of \cold~or \prac~concepts. These are also results that if you gave me 30 minutes to an hour, I should be able to derive.
        \par \know: I can recall this theorem without searching it up or referencing any text.
    }%
}
\section{Random Variables and Their Distributions}
\begin{itemize}[nosep]
    \item[\cold] \textit{Define} a probability space.
    \item[\cold] \textit{Define} a random variable
    \item[\cold] \textit{Define} a probability distribution of a random variable (or the probability distribution induced by a random variable).
    \item[\cold] \textit{Define} the expectation of a random variable. 
    \item[\cold] \textit{Define} the variance of a random variable. \textit{Prove} that $\Var{\xi} = \Exp{\xi^2} - \Exp{\xi}$.
    \item[\prac] \textit{Prove} Markov's inequality.
    \item[\prac] \textit{Prove} Chebyshev's inequality.
    \item[\cold] \textit{Define} the correlation coefficient of two random variables. 
    \item[\prac] \textit{Prove} that the correlation coefficient is less than or equal to 1. And if the correlation coefficient of $\xi_1,\xi_2$ is 1, then almost surely $\Exp{\xi_1} = a\Exp{\xi_2} + b$ for some $a,b \in \R$. 
    \item[\prac] \textit{Define} a distribution function. 
    \item[\prac] \textit{Prove} that any distribution function is right-continuous.
    \item[\cold] \textit{Prove} the Borel-Cantelli lemma. 
    \item[\prac] \textit{Prove} that both the mean and variance of the Poisson distribution, $\Pois(\lambda)$, is $\lambda$. 
    \item[\elegant] Let $x_1,x_2$ be two integers randomly and independently chosen from the set $\{1,2,\ldots,n\}$ according to the uniform distribution. Denote this probability measure as $\Pr^n$. What is the space of elementary outcomes? Let $A^n$ be the probability that the two numbers are coprime. What is $\lim_{n\to\infty} \Pr^n[A^n]$. 
    \item[\elegant] Suppose there are $n$ letters addressed to $n$ different people, and $n$ envelopes with addresses. The letters are mixed and then randomly placed into the envelopes. Find the probability that at least one letter is in the correct envelope. Find the limit of this probability as $n\to\infty$.
\end{itemize}
\section{Sequences of Independent Trials}
\begin{itemize}[nosep]
    \item[\cold] \textit{Define} a homogeneous sequence of independent random variables. 
    \item[\cold] \textit{Prove} the law of large numbers.
    \item[\know] \textit{Prove} the MacMillan Theorem.
    \item[\elegant] \textit{Prove} using Bernstein polynomials, Weierstrass' Theorem.
    \item[\prac] \textit{Prove} the Glivenko-Cantelli Theorem.
    \item[\elegant] \textit{Prove} the Poisson Limit Theorem.
\end{itemize}
\section{Lebesgue Integral and Mathematical Expectation}
\begin{itemize}[nosep]
    \item[\cold] \textit{Define} the Riemann-Stieltjes integral.
    \item[\cold] \textit{Define} almost surely convergence. 
    \item[\cold] \textit{Define} convergence in measure. 
    \item[\cold] \textit{Define} $L^p$ convergence.
\end{itemize}

\section{Conditional Probabilities and Independence}
\begin{itemize}[nosep]
    \item[\cold] \textit{Define} conditional probability.
    \item[\cold] \textit{Prove} Bayes' Theorem.
    \item[\cold] \textit{Define} independence of two or more events, $\sigma$-algebras, and random variables. 
    \item[\prac] \textit{Prove} if two random variables are independent, then the expectation of their product is equal to the product of their expectations. 
    \item[\cold] \textit{Define} a $\pi$-system and Dynkin system.
    \item[\prac] Let $\xi_1,\ldots,\xi_n$ be independent random variables, $m_1+\ldots+m_k= n$ and $f_1,\ldots,f_k$ be measurable functions of $m_1,\ldots,m_k$ variables respectively. Then, \textit{prove} the random variables $f_1(\xi_1,\ldots,\xi_{m_1}),$ $\ldots, f_k(\xi_{m_1+\ldots+m_{k-1}+1},\ldots,\xi_n)$ are independent. 
\end{itemize}

\section{Markov Chains with a Finite Number of States}
\begin{itemize}[nosep]
    \item[\cold] \textit{Define} a stochastic matrix. 
    \item[\cold] \textit{Prove} that a product of stochastic matrices is a stochastic matrix. 
    \item[\cold] \textit{Define} a Markov chain.
    \item[\cold] \textit{Define} a homogeneous Markov chain.
    \item[\cold] \textit{Define} an ergodic stochastic matrix. \textit{Define} a homogeneous, ergodic, Markov chain.
    \item[\cold] \textit{Define} a stationary distribution for a matrix of transition probabilities. 
\end{itemize}

\section{Random Walks on the Lattice $\Z^d$}
\begin{itemize}[nosep]
    \item[\cold] \textit{Define} a random walk on $\Z^d$. 
    \item[\cold] \textit{Define} a spatially homogeneous random walk.
    \item[\know] What is \textit{Polya's Theorem}?
    \item[\prac] \textit{Prove} the reflection principle.
\end{itemize}

\end{document}
